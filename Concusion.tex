%% ----------------------------------------------------------------
%% Chapter — Conclusion (no sections)
%% ----------------------------------------------------------------
\chapter{Conclusion}\label{Chapter:Conclusion}

This dissertation studied how \emph{path-wise} information extracted by signature methods can detect market regimes in a strict walk–forward setting and improve a signature-based lead--lag hedge. We started from daily BTC data from \sampleStart{} to\sampleEnd{} and built backward-looking regime labels with a signature–kernel MMD on rolling path groups and combined them with a rolling signature lead--lag matrix across assets. We then overlaid the baseline hedge (long followers / short leaders) with state-conditioned basket selection keyed to the BTC anchor, while preserving the hedge structure.

The overlay raises return per unit risk at the same market participation. Cumulative return increases from 59.53\% to 132.78\%. Sharpe rises from 0.58 to 1.46 and Sortino from 0.90 to 2.46, while annualised volatility falls from 38.09\% to 20.06\% (Table~\ref{tab:main}). Drawdowns are shallower and shorter: the maximum improves from --35.32\% to --14.15\%, and underwater time halves from 495 to 238 days (Table~\ref{tab:diag_dd}). Year-by-year results show three wins out of four (2021, 2023, 2024 YTD) and a mild shortfall in 2022 (Table~\ref{tab:eoy}). Across strict walk–forward runs, shorter windows with more paths perform best; \(n_{\text{steps}}{=}5\) with \(n_{\text{paths}}\in\{16,20\}\) dominates neighbouring settings (Table~\ref{tab:top5_sharpe_params}). Statistical checks are consistent with a cautious reading: the top configuration is significant versus zero and its block-bootstrap CIs exclude zero (Table~\ref{tab:sig_compact}), whereas we do not reject no outperformance relative to the baseline (Table~\ref{tab:wrc_spa}). Costs matter but do not erase the edge at realistic frictions: the baseline breaks even around 5–10 bps/side; the top overlay remains attractive at 5–10 bps and degrades beyond 25 bps (Tables~\ref{tab:baseline_tc}–\ref{tab:tc_sensitivity_reduced}).

The mechanism is relatively simple. The overlay does not lever up, extend holding time, or chase beta. It \emph{routes} the same hedge through regime information. In bull states it favours names that co-move with the anchor; in bear states it flips signs; in neutral states it scales. Gross exposure stays similar, but composition rotates with the state. This reduces whipsaws, trims left-tail losses, and speeds recovery.

Methodologically and empirically, the work contributes four elements: (i) a strict walk–forward regime detector based on signature–kernel MMD with medoid mapping to bull/neutral/bear labels; (ii) a signature-based lead--lag matrix that captures directional, order-sensitive dependence and a transparent baseline hedge; (iii) a \emph{regime-aware overlay} that preserves the hedge while gating baskets by an anchor-signed relation; and (iv) a reproducible evaluation protocol with unified calendar alignment, signal-to-execution timing, turnover-based costs, and diagnostics for drawdowns, yearly performance, and parameter sensitivity.

The conclusions are bounded. The sample is daily and crypto-centric over \sampleStart{}–\sampleEnd{}; external validity to other periods or asset classes is not guaranteed. A single BTC anchor can miss leadership outside crypto. Several hyperparameters are fixed (kernel scale, order, regularisation), and unbiased MMD raises variance in small training windows. Rolling votes reduce label jitter but add lag near turning points. Row-mean leadership compresses pairwise uncertainty, and lag maximisation can overfit short windows. Frictions—fees, funding/borrow, impact—are modelled parsimoniously; net results depend on venue microstructure. Finally, model selection risk remains despite a compact grid; the WRC/SPA outcome underlines that caution.

Promising extensions follow naturally. Multi-anchor or hierarchical regimes could generalise the design beyond BTC. Probabilistic (soft) states would allow confidence-weighted scaling of exposure and trades. Cost-aware construction—turnover penalties, participation caps, and state-dependent rebalance spacing—could harden net performance. Kernel approximations and streaming clustering would improve scalability. Broader horizons (pre-2021 history, intraday data) would test temporal and frequency robustness.

In summary, path-wise regime information can be combined with directional lead--lag structure to produce a \emph{regime-aware hedge} that is more resilient and capital efficient than its unconditional counterpart. The approach is modular: path features, regime mapping, and portfolio rules can be swapped independently. This makes the framework a practical template for bringing modern geometric time-series tools into risk-managed trading.
