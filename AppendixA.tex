%% ----------------------------------------------------------------
%% AppendixA.tex
%% ---------------------------------------------------------------- 
\begin{longtable}{@{}rll@{}}
\caption{Asset list with full names and tickers}\label{tab:assets}\\
\toprule
No. & Full name & Ticker \\
\midrule
\endfirsthead
\toprule
No. & Full name & Ticker \\
\midrule
\endhead
\bottomrule
\endfoot

1  & Oasis Network        & \texttt{ROSE} \\
2  & Ankr                 & \texttt{ANKR} \\
3  & VeChain              & \texttt{VET} \\
4  & NEAR Protocol        & \texttt{NEAR} \\
5  & Ethereum             & \texttt{ETH} \\
6  & EOS                  & \texttt{EOS} \\
7  & BakeryToken          & \texttt{BAKE} \\
8  & The Graph            & \texttt{GRT} \\
9  & Reef                 & \texttt{REEF} \\
10 & Injective            & \texttt{INJ} \\
11 & Filecoin             & \texttt{FIL} \\
12 & Polygon              & \texttt{MATIC} \\
13 & Bitcoin Cash         & \texttt{BCH} \\
14 & IOST                 & \texttt{IOST} \\
15 & Chromia              & \texttt{CHR} \\
16 & MultiversX           & \texttt{EGLD} \\
17 & Hedera               & \texttt{HBAR} \\
18 & Zilliqa              & \texttt{ZIL} \\
19 & Algorand             & \texttt{ALGO} \\
20 & Dent                 & \texttt{DENT} \\
21 & Dash                 & \texttt{DASH} \\
22 & My Neighbor Alice    & \texttt{ALICE} \\
23 & IOTA                 & \texttt{IOTA} \\
24 & Chainlink            & \texttt{LINK} \\
25 & Tezos                & \texttt{XTZ} \\
26 & Loopring             & \texttt{LRC} \\
27 & Harmony              & \texttt{ONE} \\
28 & Solar                & \texttt{SXP} \\
29 & Kava                 & \texttt{KAVA} \\
30 & Axie Infinity        & \texttt{AXS} \\
31 & Cardano              & \texttt{ADA} \\
32 & Solana               & \texttt{SOL} \\
33 & Ontology             & \texttt{ONT} \\
34 & Ethereum Classic     & \texttt{ETC} \\
35 & Decentraland         & \texttt{MANA} \\
36 & Synthetix            & \texttt{SNX} \\
37 & Zcash                & \texttt{ZEC} \\
38 & Conflux              & \texttt{CFX} \\
39 & Yearn Finance        & \texttt{YFI} \\
40 & Waves                & \texttt{WAVES} \\
41 & Litecoin             & \texttt{LTC} \\
42 & Chiliz               & \texttt{CHZ} \\
43 & Stellar              & \texttt{XLM} \\
44 & COTI                 & \texttt{COTI} \\
45 & Polkadot             & \texttt{DOT} \\
46 & OMG Network          & \texttt{OMG} \\
47 & SushiSwap            & \texttt{SUSHI} \\
48 & Fantom               & \texttt{FTM} \\
49 & BNB                  & \texttt{BNB} \\
50 & Uniswap              & \texttt{UNI} \\
51 & Stacks               & \texttt{STX} \\
52 & THORChain            & \texttt{RUNE} \\
53 & Theta Network        & \texttt{THETA} \\
54 & Holo                 & \texttt{HOT} \\
55 & 1inch                & \texttt{1INCH} \\
56 & Fetch.ai             & \texttt{FET} \\
57 & Kusama               & \texttt{KSM} \\
58 & Smooth Love Potion   & \texttt{SLP} \\
59 & Curve DAO Token      & \texttt{CRV} \\
60 & IoTeX                & \texttt{IOTX} \\
61 & Bitcoin              & \texttt{BTC} \\
62 & Avalanche            & \texttt{AVAX} \\
63 & Enjin Coin           & \texttt{ENJ} \\
64 & PancakeSwap          & \texttt{CAKE} \\
65 & XRP                  & \texttt{XRP} \\
66 & TRON                 & \texttt{TRX} \\
67 & Cosmos               & \texttt{ATOM} \\
68 & Aave                 & \texttt{AAVE} \\
69 & Dogecoin             & \texttt{DOGE} \\
70 & Neo                  & \texttt{NEO} \\
71 & The Sandbox          & \texttt{SAND} \\
72 & Qtum                 & \texttt{QTUM} \\
\end{longtable}

% Notation
% D in R^{G x G}, symmetric; t_i end times; raw boundary t_train^raw; horizon H
% Safe boundary: t_train^safe = t_train^raw - H
\iffalse
\textbf{Training set:}\quad
I_{\text{train}} = \{\, i : t_i \le t_{\text{train}}^{\text{safe}} \,\}.

\textbf{Clustering \& medoids:}\quad
\text{Cluster } D[I_{\text{train}}, I_{\text{train}}] \to \{C_k\}_{k=1}^K,\quad
m_k = \arg\min_{j \in C_k} \sum_{i \in C_k} D_{ij}.

\textbf{Forward return (H steps) for } i \in C_k:\quad
r_H(i) = \frac{P(t_i + H) - P(t_i)}{P(t_i)}.

\textbf{Discrete direction:}\quad
\mathrm{dir}(i) =
\begin{cases}
\text{Bull}, & r_H(i) \ge \tau_{\text{pos}},\\
\text{Bear}, & r_H(i) \le -\tau_{\text{neg}},\\
\text{Sideways}, & \text{otherwise.}
\end{cases}

\textbf{Cluster semantic (majority vote):}\quad
L_k \in \{\text{Bear, Sideways, Bull}\}
\text{ from } \{\mathrm{dir}(i): i \in C_k\}.
\text{ (Tie-break by sign of } \overline{r_H} \text{ or medoid’s label.)}

\textbf{Prototype dictionary:}\quad
\mathcal{M} = \{(m_k, L_k)\}_{k=1}^K.

\textbf{Assignment on all groups:}\quad
c(g) = \arg\min_k D_{g, m_k}, \qquad
\widehat{\mathrm{dir}}(g) = L_{c(g)}.

\textbf{Rolling vote (optional):}\quad
\widetilde{\mathrm{dir}}(g) = \operatorname{mode}\big(\widehat{\mathrm{dir}}(g-k_{\text{vote}}+1),\dots,\widehat{\mathrm{dir}}(g)\big).

\textbf{Daily mapping:}\quad
\text{Expand group labels over their date spans to obtain a daily series.}

\textbf{No-leakage:}\quad
\text{Only use } i \text{ with } t_i + H \le t_{\text{train}}^{\text{raw}}.
\text{ Equivalently, } t_{\text{train}}^{\text{safe}} = t_{\text{train}}^{\text{raw}} - H.
\fi

\section{Signature Theory}

The signature of a path is a mathematical object that originates from rough path theory (Lyons, 1998). 
Given a $d$-dimensional continuous path $X : [0,T] \to \mathbb{R}^d$, the signature is defined as the infinite collection of \emph{iterated integrals} of the path: 
\[
S(X) = \left( 1, S^{(1)}(X), S^{(2)}(X), S^{(3)}(X), \dots \right),
\]
where the $k$-th level signature term is
\[
S^{(k)}(X) = \int_{0 < t_1 < \cdots < t_k < T} dX_{t_1} \otimes dX_{t_2} \otimes \cdots \otimes dX_{t_k}.
\]

At the first level ($k=1$), the signature reduces to the net increment of the path:
\[
S^{(1)}(X) = X_T - X_0.
\]
At the second level, we obtain quadratic information that encodes the order of movements:
\[
S^{(2)}(X) = \int_{0 < t_1 < t_2 < T} dX_{t_1} \otimes dX_{t_2},
\]
which captures how one coordinate of the path leads or lags another. Higher-order terms generalise this to more complex dependencies.

Two properties make the signature especially powerful. First, \textbf{Chen’s identity} guarantees that the signature of a concatenated path can be decomposed into signatures of its segments, making it suitable for sequential data. Second, under mild conditions, the \textbf{signature uniquely characterises the path} up to so-called ``tree-like'' equivalence, meaning that it retains essentially all the information of the original trajectory (Lyons, 1998; Hambly and Lyons, 2010). 

Since the full signature is infinite-dimensional, in practice one truncates it at depth $m$. 
The dimension of the truncated signature grows polynomially:
\[
\dim S^{\leq m}(X) = \sum_{k=0}^m d^k,
\]
which means that for even moderate $d$ and $m$, the feature space becomes high-dimensional. This motivates careful choices of truncation level and path embedding.


Financial time series such as asset prices are inherently sequential and path-dependent. Traditional summary statistics---such as mean, variance, or simple autocorrelations---capture only coarse properties of the distribution. In contrast, signature features encode the order and interaction of movements at multiple scales, making them particularly suitable for detecting subtle temporal structures.  
For example, second-order signature terms can reflect lead--lag effects between variables, while higher-order terms can characterise nonlinear dependencies that may indicate regime shifts.  
This ability to represent paths beyond linear summary statistics makes the signature method a natural candidate for regime detection in financial markets.

\subsection{Maximum Mean Discrepancy (MMD)}

Once features are extracted, the problem of regime detection becomes a problem of identifying distributional changes over time. To this end, we adopt the \textbf{Maximum Mean Discrepancy (MMD)} as a non-parametric two-sample test (Gretton et al., 2012).  

Given two sets of observations $X = \{x_i\}_{i=1}^n$ and $Y = \{y_j\}_{j=1}^m$ in feature space $\mathcal{X}$, the squared MMD between their distributions $P$ and $Q$ is defined as:
\[
\mathrm{MMD}^2(P,Q;\mathcal{H}) = \sup_{f \in \mathcal{H}, \|f\|_{\mathcal{H}} \leq 1} \left( \mathbb{E}_{x \sim P}[f(x)] - \mathbb{E}_{y \sim Q}[f(y)] \right)^2,
\]
where $\mathcal{H}$ is a reproducing kernel Hilbert space (RKHS).  

In practice, with a kernel function $k(\cdot, \cdot)$, the unbiased empirical estimate is:
\[
\widehat{\mathrm{MMD}}^2(X,Y) = \frac{1}{n(n-1)} \sum_{i \neq i'} k(x_i, x_{i'}) + \frac{1}{m(m-1)} \sum_{j \neq j'} k(y_j, y_{j'}) - \frac{2}{nm} \sum_{i,j} k(x_i, y_j).
\]

If $\widehat{\mathrm{MMD}}^2$ is large, it indicates that the two sets of observations likely come from different distributions, i.e., a regime shift. The choice of kernel (e.g., Gaussian RBF) controls the sensitivity to local vs. global changes.



\subsection{Construction of Distance Matrices}

To perform clustering, we require a distance measure between different segments of the BTC price path. 
The first step is \textbf{subpath extraction}. The original time series is divided into overlapping subpaths of fixed length $n_{\text{steps}}$. Each subpath represents the local dynamics of the market over a short horizon, ensuring that temporal dependencies are preserved.

Next, we introduce a \textbf{grouping stage}. Instead of treating each subpath individually, we aggregate $n_{\text{paths}}$ consecutive subpaths into groups. This reduces the effect of noise and yields more stable representations of local behaviour. Each group can be regarded as a local distribution of path features.

Finally, we compute \textbf{pairwise distances} between groups. After transforming each path using the signature representation, we employ a Maximum Mean Discrepancy (MMD)–based metric to quantify the distributional difference between two groups. The outcome is a symmetric distance matrix $D \in \mathbb{R}^{G \times G}$, where $G$ denotes the number of groups. This distance matrix serves as the input for subsequent clustering, enabling the identification of distinct market regimes.